\documentclass[t]{beamer}
\usepackage{multicol}

\usetheme{default}
\setbeamertemplate{navigation symbols}{}

%\mode<presentation>
%{
%  \usetheme{Warsaw}
%  % or ...
%
%  \setbeamercovered{transparent}
%  % or whatever (possibly just delete it)
%}


\usepackage[english]{babel}
% or whatever

\usepackage[latin1]{inputenc}
% or whatever

\usepackage{times}
\usepackage[T1]{fontenc}
% Or whatever. Note that the encoding and the font should match. If T1
% does not look nice, try deleting the line with the fontenc.


%\usepackage{verbatim}
\usepackage{colortbl}

 %\newrgbcolor{hilight}{0.000 1.000 0.000}   % green
 %\xdefinecolor{hilight}{rgb}{0.000 1.000 0.000} % green
 %\newcommand{\hilight}{\color{blue}}
 %\xdefinecolor{MyGreen}{rgb}{0.000, 0.500, 0.000} % green
 \colorlet{myred}{red!90!black} % slightly darker than normal red
 \colorlet{myblue}{cyan}        % plain cyan works; might consdier darkening
 \colorlet{mygreen}{green!70!black} % slightly darker than normal green
 %\newcommand{\nonliteral}{\color{MyGreen}}

\title{Merging in Git}

\author{Elijah Newren}
\institute{}

\date{}

\subject{Talks}
% This is only inserted into the PDF information catalog. Can be left
% out.



% If you have a file called "university-logo-filename.xxx", where xxx
% is a graphic format that can be processed by latex or pdflatex,
% resp., then you can add a logo as follows:

% \pgfdeclareimage[height=0.5cm]{university-logo}{university-logo-filename}
% \logo{\pgfuseimage{university-logo}}



% If you wish to uncover everything in a step-wise fashion, uncomment
% the following command:

%\beamerdefaultoverlayspecification{<+->}


\begin{document}

\begin{frame}
  \titlepage
\end{frame}

%%%%%%%%%%%%%%%%%%%%%%%%%%%%%%%%%%%%%%%%%%%%%%%%%%%%%%%%%%%%%%%%%%%%%%%%%%

\section{Background}

\begin{frame}
  \frametitle{Three-way content merge}

  \begin{multicols*}{2}
    File from branch A:\\
    {\footnotesize\texttt{%
    ...                             \\
    speak\_like\_a\_pirate(arrrgs); \\
    \only<3->{\color{myred}}explore\_sea(aye, matey);\only<3->{\color{black}}\\
    shiver(me.timbers);             \\
    ...                             \\
    }}
    \columnbreak
    \pause
    Same file from branch B:\\
    {\footnotesize\texttt{%
    ...                             \\
    speak\_like\_a\_pirate(arrrgs); \\
    \only<3->{\color{myred}}explore\_sea(me.love[0]);\only<3->{\color{black}}\\
    shiver(me.timbers);             \\
    ...
    }}
  \end{multicols*}%

  \only<4->{
    Correct merge depends on the version in the merge base:\\
    {\footnotesize\texttt{%
      speak\_like\_a\_pirate(arrrgs);\\
      \only<4-5>{{\color{myblue}?????}}
      \only<6-7>{{\color{myblue}explore\_sea(aye, matey);}}
      \only<8-9>{{\color{myblue}explore\_sea(me.love[0]);}}
      \only<10-11>{{\color{myblue}explore\_sea(plus, plus);}}
      \\
      shiver(me.timbers);\\
    }}

    \only<5->{
    \vspace*\baselineskip
    \only<5>{Which we need to know to determine the merge:}
    \only<6->{Which results in the following merge:}\\
    {\footnotesize\texttt{%
      speak\_like\_a\_pirate(arrrgs);\\
      \only<6,8,10>{\qquad}
      \only<5>{{\color{mygreen}?????}}
      \only<7>{{\color{mygreen}explore\_sea(me.love[0]);}}
      \only<9>{{\color{mygreen}explore\_sea(aye, matey);}}
      \only<11>{{\color{mygreen}
      <{}<{}<{}<{}<{}<{}< HEAD  \\
      explore\_sea(aye, matey); \\
      =======\\
      explore\_sea(me.love[0]); \\
      >{}>{}>{}>{}>{}>{}> branchB
      }}
      \\
      shiver(me.timbers);\\
    }}
    }
  }

\end{frame}

%%%%%%%%%%%%%%%%%%%%%%%%%%%%%%%%%%%%%%%%%%%%%%%%%%%%%%%%%%%%%%%%%%%%%%%%%%

\begin{frame}
  \frametitle{Three-way content merge, cont.}

  git's sha1sum of individual files can be used for a shorthand:\\
    {\footnotesize\texttt{%
    \hspace*{0.6em} path\\
    1:\ sha1sum(orig:path)\\
    2:\ sha1sum(A:path)\\
    3:\ sha1sum(B:path)\\[\baselineskip]
    }}

  \pause

  For example (using shortened shas here):\\
    {\footnotesize\texttt{%
    \hspace*{0.6em} pirate.java\\
    1:\ 5ca1ab1e\\
    2:\ f005ba11\\
    3:\ b0a710ad\\[\baselineskip]
    }}

  \pause

  Where the ordering is as follows:
  {\footnotesize
  \begin{enumerate}
    \item merge base
    \item HEAD (branch checked out before running merge)
    \item other branch (the argument you passed to merge)
  \end{enumerate}
  }

  \pause

  \vspace*{0.5\baselineskip}
  git makes these accessible...

\end{frame}

\end{document}
