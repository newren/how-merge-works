\documentclass[t]{beamer}
\usepackage{multicol}

\usetheme{default}
\setbeamertemplate{navigation symbols}{}

%\mode<presentation>
%{
%  \usetheme{Warsaw}
%  % or ...
%
%  \setbeamercovered{transparent}
%  % or whatever (possibly just delete it)
%}


\usepackage[english]{babel}
% or whatever

\usepackage[latin1]{inputenc}
% or whatever

\usepackage{times}
\usepackage[T1]{fontenc}
% Or whatever. Note that the encoding and the font should match. If T1
% does not look nice, try deleting the line with the fontenc.


%\usepackage{verbatim}
\usepackage{colortbl}

 %\newrgbcolor{hilight}{0.000 1.000 0.000}   % green
 %\xdefinecolor{hilight}{rgb}{0.000 1.000 0.000} % green
 %\newcommand{\hilight}{\color{blue}}
 %\xdefinecolor{MyGreen}{rgb}{0.000, 0.500, 0.000} % green
 \colorlet{myred}{red!90!black} % slightly darker than normal red
 \colorlet{myblue}{cyan}        % plain cyan works; might consdier darkening
 \colorlet{mygreen}{green!70!black} % slightly darker than normal green
 %\newcommand{\nonliteral}{\color{MyGreen}}

 \newenvironment{myitemize}
                {
                  \begin{itemize}
                    \setlength{\itemsep}{0pt}
                    \setlength{\parskip}{-0.125\baselineskip}
                    \setlength{\parsep} {0pt}
                }%
                {\end{itemize}}

\title{Merging in Git}

\author{Elijah Newren}
\institute{}

\date{}

\subject{Talks}
% This is only inserted into the PDF information catalog. Can be left
% out.



% If you have a file called "university-logo-filename.xxx", where xxx
% is a graphic format that can be processed by latex or pdflatex,
% resp., then you can add a logo as follows:

% \pgfdeclareimage[height=0.5cm]{university-logo}{university-logo-filename}
% \logo{\pgfuseimage{university-logo}}



% If you wish to uncover everything in a step-wise fashion, uncomment
% the following command:

%\beamerdefaultoverlayspecification{<+->}


\begin{document}

\begin{frame}
  \titlepage
\end{frame}

\begin{frame}{Overview}
  \tableofcontents
\end{frame}

%%%%%%%%%%%%%%%%%%%%%%%%%%%%%%%%%%%%%%%%%%%%%%%%%%%%%%%%%%%%%%%%%%%%%%%%%%

\section{Three-way content merge}

\begin{frame}
  \frametitle{Three-way content merge}

  \begin{multicols}{2}
    File from branch A:\\
    {\footnotesize\texttt{%
    ...                             \\
    speak\_like\_a\_pirate(arrrgs); \\
    \only<3->{\color{myred}}explore\_sea(aye, matey);\only<3->{\color{black}}\\
    shiver(me.timbers);             \\
    ...                             \\
    }}
    \columnbreak
    \pause
    Same file from branch B:\\
    {\footnotesize\texttt{%
    ...                             \\
    speak\_like\_a\_pirate(arrrgs); \\
    \only<3->{\color{myred}}explore\_sea(me.love[0]);\only<3->{\color{black}}\\
    shiver(me.timbers);             \\
    ...
    }}
  \end{multicols}%

  \only<4->{
    Correct merge depends on the version in the merge base:\\
    {\footnotesize\texttt{%
      speak\_like\_a\_pirate(arrrgs);\\
      \only<4-5>{{\color{myblue}?????}}
      \only<6-7>{{\color{myblue}explore\_sea(aye, matey);}}
      \only<8-9>{{\color{myblue}explore\_sea(me.love[0]);}}
      \only<10-11>{{\color{myblue}explore\_sea(plus, plus);}}
      \\
      shiver(me.timbers);\\
    }}

    \only<5->{
    \vspace*\baselineskip
    \only<5>{Which we need to know to determine the merge:}
    \only<6->{Which results in the following merge:}\\
    {\footnotesize\texttt{%
      speak\_like\_a\_pirate(arrrgs);\\
      \only<6,8,10>{\qquad}
      \only<5>{{\color{mygreen}?????}}
      \only<7>{{\color{mygreen}explore\_sea(me.love[0]);}}
      \only<9>{{\color{mygreen}explore\_sea(aye, matey);}}
      \only<11>{{\color{mygreen}
      <{}<{}<{}<{}<{}<{}< HEAD  \\
      explore\_sea(aye, matey); \\
      =======\\
      explore\_sea(me.love[0]); \\
      >{}>{}>{}>{}>{}>{}> branchB
      }}
      \\
      shiver(me.timbers);\\
    }}
    }
  }

\end{frame}

%%%%%%%%%%%%%%%%%%%%%%%%%%%%%%%%%%%%%%%%%%%%%%%%%%%%%%%%%%%%%%%%%%%%%%%%%%

\begin{frame}
  \frametitle{Three-way content merge, cont.}

  git's sha1sum of individual files can be used for a shorthand:\\
    {\footnotesize\texttt{%
    \hspace*{0.6em} path\\
    1:\ sha1sum(orig:path)\\
    2:\ sha1sum(A:path)\\
    3:\ sha1sum(B:path)\\[\baselineskip]
    }}

  \pause

  For example (using shortened shas here):\\
    {\footnotesize\texttt{%
    \hspace*{0.6em} angryp.swine-latin\\
    1:\ 5ca1ab1e\\
    2:\ f005ba11\\
    3:\ b0a710ad\\[\baselineskip]
    }}

  \pause

  Where the ordering is as follows:
  {\footnotesize
  \begin{enumerate}
    \item merge base
    \item HEAD (branch checked out before running merge)
    \item other branch (the argument you passed to merge)
  \end{enumerate}
  }

  \pause

  \vspace*{0.5\baselineskip}
  git makes these accessible...

\end{frame}

%%%%%%%%%%%%%%%%%%%%%%%%%%%%%%%%%%%%%%%%%%%%%%%%%%%%%%%%%%%%%%%%%%%%%%%%%%

\section{Helpful git commands}
\begin{frame}
  \frametitle{Helpful commands}

  Pro-tip: You can ask git to check if there are conflict markers or
  whitespace errors:\\[0.25\baselineskip]
    {\scriptsize\texttt{%
    \$ git diff -{}-check\\
    angryp.swine-latin:2: leftover conflict marker\\
    angryp.swine-latin:4: leftover conflict marker\\
    angryp.swine-latin:6: leftover conflict marker\\
    }}

\end{frame}

%%%%%%%%%%%%%%%%%%%%%%%%%%%%%%%%%%%%%%%%%%%%%%%%%%%%%%%%%%%%%%%%%%%%%%%%%%

\begin{frame}
  \frametitle{Helpful commands}

  \only<1-8>{
  Getting details about which files are conflicted:\\
    {\footnotesize\texttt{%
    \$ git ls-files -u \\
    100644 41e3dc22a02a972d0d42 1 \qquad angryp.swine-latin\\
    100644 f185132ce93bf3e453b8 2 \qquad angryp.swine-latin\\
    100644 b506e78238513afdfbb0 3 \qquad angryp.swine-latin\\[\baselineskip]
    }}
  }

  \only<2-3>{
    Viewing other versions:\\
      {\footnotesize\texttt{%
      \$ git show :stage:filename \\
      \$ git show sha1sum \\[\baselineskip]
      }}

    \only<3>{
    Example:\\
      {\footnotesize\texttt{%
      \$ git show :2:angryp.swine-latin \\
      speak\_like\_a\_pirate(arrrgs);\\
      explore\_sea(aye, matey);\\
      shiver(me.timbers);\\[\baselineskip]
      }}
    }
  }

  \only<4-5>{
    Diffing against other versions:\\
      {\footnotesize\texttt{%
      \$ git diff [-{}-base|-{}-ours|-{}-theirs] [filename] \\
      }}

    \only<5>{
    Example:\\
      {\footnotesize\texttt{%
      \$ git diff -{}-base \\
      ...\\
      @@ -1,3 +1,7 @@\\
      \ speak\_like\_a\_pirate(arrrgs);\\
      {\color{myred}-explore\_sea(plus, plus);}\\
      {\color{mygreen}%
      +<{}<{}<{}<{}<{}<{}< HEAD\\
      +explore\_sea(aye, matey);\\
      +=======\\
      +explore\_sea(me.love[0]);\\
      +>{}>{}>{}>{}>{}>{}> branchB}\\
      \ shiver(me.timbers);\\[\baselineskip]
      }}
    }
  }

  \only<6-8>{
    Ovewriting with different versions:\\
      {\footnotesize\texttt{%
      \$ git checkout [-{}-ours|-{}-theirs] <filename> \\
      \$ git checkout [-{}-merge|-m|-{}-conflict=diff3] <filename>
          \\[0.5\baselineskip]
      }}

    \only<7>{
    Example:\\
      {\footnotesize\texttt{%
      \$ git checkout -{}-ours angryp.swine-latin \\
      \$ cat angryp.swine-latin \\
      speak\_like\_a\_pirate(arrrgs);\\
      explore\_sea(aye, matey);\\
       shiver(me.timbers);\\[\baselineskip]
      }}
    }

    \only<8>{
    Example:\\
      {\footnotesize\texttt{%
      \$ git checkout -{}-conflict=diff3 angryp.swine-latin\\
      \$ cat angryp.swine-latin \\
      speak\_like\_a\_pirate(arrrgs);\\
      <{}<{}<{}<{}<{}<{}< ours\\
      explore\_sea(aye, matey);\\
      ||||||| base\\
      explore\_sea(plus, plus);\\
      =======\\
      explore\_sea(me.love[0]);\\
      >{}>{}>{}>{}>{}>{}> theirs\\
      shiver(me.timbers);\\[\baselineskip]
      }}
    }
  }

  \only<9-12>{
    \only<9-11>{
    Can look for commits which touched conflicted files:\\[0.5\baselineskip]
      {\scriptsize\texttt{%
      \$ git log HEAD...MERGE\_HEAD -{}- \textbackslash\\
          \hspace*{10em} `git ls-files -u | awk \{print\textbackslash\$4\}`\\
      {\color{brown}commit 95d844d711a8ba9fae97 (}{\color{cyan}HEAD}, {\color{cyan}branchA}{\color{brown})}\\
      Author:\ Cap'n Blackbeard <black@beard.pirate>\\
      Date:\   Tue May 22 13:53:03 2018 -0700\\[\baselineskip]
      \qquad\!    Aye, aye\\[\baselineskip]
      {\color{brown}commit 34fa04c4a1962cf949b3 (}{\color{cyan}branchB}{\color{brown})}\\
      Author:\ Cap'n Whitebeard <white@beard.pirate>\\
      Date:\   Tue May 22 13:52:48 2018 -0700\\[\baselineskip]
      \qquad\!    Me first love\\[\baselineskip]
      }}
    }

    \only<10-11>{
      But there's an equivalent simple shorthand:\\
      {\scriptsize\texttt{%
      \$ git log -{}-merge\\[\baselineskip]
      }}
    }

    \only<11>{
      Which can be handy in combination with other flags, e.g.:\\
      {\scriptsize\texttt{%
      \$ git log -{}-merge -p -{}-oneline -{}-left-right\\
      }}
    }

    \only<12>{
      Using handy -{}-merge flag to log:\\[0.25\baselineskip]
      {\scriptsize\texttt{%
      \$ git log -{}-merge -p -{}-oneline -{}-left-right\\[0.5\baselineskip]
      {\color{blue}\textbf{<}} {\color{brown}95d844d (}{\color{cyan}HEAD}, {\color{cyan}branchA}{\color{brown})} Aye, aye\\
      diff --git a/angryp.swine-latin b/angryp.swine-latin\\
      index 41e3dc2..f185132 100644\\
      --- a/angryp.swine-latin\\
      +++ b/angryp.swine-latin\\
      @@ -1,3 +1,3 @@\\
      \ speak\_like\_a\_pirate(arrrgs);\\
      {\color{myred}-explore\_sea(plus, plus);}\\
      {\color{mygreen}+explore\_sea(aye, matey);}\\
      \ shiver(me.timbers);\\[0.5\baselineskip]
      {\color{blue}\textbf{>}} {\color{brown}34fa04c (}{\color{cyan}branchB}{\color{brown})} Me first love\\
      diff --git a/angryp.swine-latin b/angryp.swine-latin\\
      index 41e3dc2..b506e78 100644\\
      --- a/angryp.swine-latin\\
      +++ b/angryp.swine-latin\\
      @@ -1,3 +1,3 @@\\
      \ speak\_like\_a\_pirate(arrrgs);\\
      {\color{myred}-explore\_sea(plus, plus);}\\
      {\color{mygreen}+explore\_sea(me.love[0]);}\\
      \ shiver(me.timbers);\\
      }}
    }

  }

\end{frame}

%%%%%%%%%%%%%%%%%%%%%%%%%%%%%%%%%%%%%%%%%%%%%%%%%%%%%%%%%%%%%%%%%%%%%%%%%%

\begin{frame}
  \frametitle{Helpful commands, summarized}

  Checking for remaining conflict markers:\\
    {\scriptsize\texttt{%
    \$ git diff -{}-check\\[\baselineskip]
    }}

  Getting details about which files are conflicted:\\
    {\footnotesize\texttt{%
    \$ git ls-files -u \\[\baselineskip]
    }}

  Viewing other versions:\\
    {\footnotesize\texttt{%
    \$ git show :stage:filename \\[\baselineskip]
    }}

  Diffing against other versions:\\
    {\footnotesize\texttt{%
    \$ git diff [-{}-base|-{}-ours|-{}-theirs] [filename(s)] \\[\baselineskip]
    }}

  Ovewriting with different versions:\\
    {\footnotesize\texttt{%
    \$ git checkout [-{}-ours|-{}-theirs] <filename> \\
    \$ git checkout [-{}-merge|-m|-{}-conflict=diff3] <filename>
          \\[\baselineskip]
    }}

  Seeing which commits touched conflicted files:\\
    {\scriptsize\texttt{%
    \$ git log -{}-merge -p -{}-left-right\\[\baselineskip]
    }}

\end{frame}

%%%%%%%%%%%%%%%%%%%%%%%%%%%%%%%%%%%%%%%%%%%%%%%%%%%%%%%%%%%%%%%%%%%%%%%%%%

\section{Caveats}
\begin{frame}
  \frametitle{Caveats}

  {\scriptsize

  Critiques/Limitations of three-way merge idea\\[-0.5\baselineskip]
  \begin{myitemize}
    \item Context region size tradeoffs%; issues:
      %* what's the correct size for the context region?  too small?  too big?
      %* example: import/include statements -- context lines irrelevant
    \item File Content
      \begin{myitemize}\scriptsize
        \item Files must be ``normal'' (text)%; counter-examples:
          %* binary
          %* symlinks
          %* submodules
        \item Assumes line-based diffing%; issues:
          %* journal, documentation, word-based articles
        \item Text-based diff ignores semantic content%; issues:
          %* one branch adds call to a function, another adds an argument to it
        \item Needs file ``encoding'' to be the same% (merge.renormalize); issues:
          %* ASCII vs. EBCDIC (yeah, sure)
          %* CR vs. CRLF (happens...)
          %* whitespace normalization?
          %* unicode normalization?
          %* other programmatic modifications -- indentation, etc.?
        \item Assumes conflict markers are distinguishable from other text
          % What if you're writing documenting, about how git-merge works?
      \end{myitemize}\vspace*{-0.3\baselineskip}
    \item Ignores intermediate history%; issues:
      %* "fix" applied on both sides, then reverted on one side because it's
      %  harmful...
      %* same fix on both sides, extra fixes on other side in same region =>
      %  conflicts
      %* however, three-way merge is hard enough for people to understand.
      %  Explaining a conflict that arises from intermediate history might
      %  become truly baffling
  \end{myitemize}\vspace*{-0.3\baselineskip}

  Applicability of three-way merge idea:\\[-0.7\baselineskip]
  \begin{myitemize}
    \item Merge-base
      \begin{myitemize}\scriptsize
      \item exists%; counter-examples:
        %* merging independent repositories -- do two-way merge
      \item unique%; counter-examples:
        %* lots; see git.git and linux.git.  unlikely with central server and
        %  merge-to-mainline workflow
        %* follow-on assumption: merge-base can be constructed by just merging
        %  the others together (counter-examples: conflicts, including crazy types)
        %* follow-on assumption: we can find something meaningful to use as a
        %  merge-base and construct it as a tree (maybe...)
      \end{myitemize}\vspace*{-0.3\baselineskip}
    \item File versions
      \begin{myitemize}\scriptsize
        \item Three versions of each file exist%; counter-examples:
          %* one version
          % * existed in base \& deleted on both sides
          % * new file unique to one side (possible D/F conflict...)
          %* two versions
          %  * didn't exist in base; add/add conflict unless identical
          %  * did exist in base: modify/delete unless unmodified
          %* more than three versions: (see combination effects)
        \item The three versions of the file have same pathname
          %* whoo, boy, this is a big can of worms
        \item If there are three versions of pathname, they're related%; counter-examples:
          %* simple rename/modify (use O,A,B notation to show modify/delete+add)
          %* one renamed away, with new file in place
          %* one file deleted, new file now in its place
          %* swap rename (a->b, b->a)
          %* (This could get really complicated, but git doesn't do break detection
          %   for merges, so not relevant right now)
          \end{myitemize}\vspace*{-0.3\baselineskip}
    \item Chunks of lines (e.g. functions) do not move between files
      %* git handles this in blame, but not for merges.  At least not yet.
  \end{myitemize}\vspace*{-0.3\baselineskip}

  Subtle assumptions:\\[-0.5\baselineskip]
  \begin{myitemize}
    \item Path for a file is writable %(whoops)
    \item No nesting of conflict markers %(ha!)
    \item Conflicts representable on disk
      %(oops: modify/delete, D/F, submodule/file,
      %submodule/directory, submodule/submodule, rename/rename(1to2), directory
      %rename detection issues)
  \end{myitemize}\vspace*{-0.3\baselineskip}

  Implementation limits:\\[-0.5\baselineskip]
  \begin{myitemize}
    \item Dirty changes
    \item Rename heuristics
    \item Writing intermediate file-merges to object store immediately
  \end{myitemize}

  }

\end{frame}

\end{document}
